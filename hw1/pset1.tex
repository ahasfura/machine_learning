\documentclass{article}

\usepackage{graphicx}
\usepackage{mathtools}
\usepackage{fullpage}

\begin{document}
\title{6.867: Homework 1}
\author{Andres Hasfura and Kathryn Evans}
\maketitle
\section{Gradient Descent}
Gradient descent works by updating an initial guess by taking a step opposite the direction of the gradient at that initial guess. Moving opposite the gradient helps  to effectively move "downhill" in terms of the function value.  The formulation of the gradient descent method is as follows
\begin{equation}
x(\tau+1) = x(\tau) - \alpha \nabla F(x(\tau))
\end{equation}

Where $ x(\tau)$ is the initial guess, $\alpha$ is the step size and $F(x)$ is the function which will be evaluated. 




\end{document}